Sea-surface-signature-based interpretation of geostrophic mesoscale ocean dynamics via space-born altimeter products has come a long way since the launch of the \href{http://en.wikipedia.org/wiki/TOPEX/Poseidon}{Topex/Poseidon} mission. In the early years, few years of data and poor spatial resolution led oceanographers to still construe the westard drifting pattern of \SSH~anomalies as Rossby waves \citep{le1993sea,Killworth1997a}.
Merging the Topex/Poseidon with the ERS-1/2 altimeter output increased the resolution by a factor of 2 \citep{Chelton2007}, revealing that most of the \SSH~variability had in fact to be accredited to non-linear mesoscale eddies. Today, the availability of a long, coherent time-span of weekly spatially consistent \SSH~data, makes global automated eddy-identification and -tracking feasible.
When interpreting censuses as such, consideration of the technical methods and thresholds used in the algorithm is important. The spectrum of geostrophic phenomena does not allow for sharp discriminations between the theoretical concepts of Rossby waves, geostrophic currents and coherent vortices.
The stringencies of the algorithm in terms of amplitude, shape, size, lifespan etc effectively define the object under investigation. Generalized statements about \textit{eddy} statistics derived from such censuses hence always hinge on the understanding of what an eddy is and how this understanding had been implemented in the algorithm.\\
By simulating a coarser spatial resolution for the \pop~data, we show that resolution of the merged \avi~product is still insufficent for exact determinations of horizontal eddy scales, especially in high latitudes.\\
By comparing determined zonal drift speeds from an analysis with a 7-day time-step to those from a 2-day time-step, we show that the success rate of automated tracking of individual eddies decreases considerably in regions of strong background-current gradients at a weekly time-step. This decreases the quality of determined drift speeds, drift trajectories and eddy life-spans. 
