\begin{abstract}
Several global mesoscale ocean-eddy-dynamics censuses were obtained via an automated, sea-surface-anomaly-based detection- and tracking-algorithm. 12 years of input sea-surface-height data were taken from a satellite-altimetry product and from a global eddy-resolving ocean model. Variables of particular interest were horizontal eddy scale and zonal eddy drift speeds. Motivation was to answer the question whether time- and space-resolutions of the altimeter-product are sufficiently fine to resolve eddy scales accurately and to allow a successful tracking of individual eddies over time. The results suggest that the $\ang{0.25}$-resolution of the merged satellite scans is insufficient to resolve realistic eddy scales in high latitudes and that a $\SI{7}{day}$ time-step, albeit generally sufficient to determine zonal eddy drift-speeds, leads to unreliable trackings in regions of very strong drift-vector gradients. The finer resolution in space and time of the model data allows for more robust determinations of eddy scales which turn out to be significantly smaller than those derived from satellite data. Zonal drift speeds are shown to be systematically slower in the model than in the satellite product. 
\end{abstract}
\newpage
