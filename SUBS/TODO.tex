

\TODO{Strong zonal skewness suggests the existence of many small, potentially erronuous values that smear the distribution of drift speeds towards an unrealisticly low mean. MEDIAN? This effect appears to be relevant in \eg the Souther Ocean, where the east-ward advection of eddies by the ACC results in a broad spectrum of drift speeds. The strong gradients in mean current also effect an abundance of eddy-merging and -splitting situations over relatively short periods of time. It is therefor difficult for the algorithm to keep track of sufficiently long-lived, coherent vortices. Especially so for large time-steps and a high age-threshold. Yet, if the minimum time-step is limited, as in the case of satellite data, a high age-threshold is necessary since short tracks with few data points in time are more prone to stem from erronuously matched contours that do not represent the actual track of a single vortex but instead represent other meso-scale noise that happened to feature sufficiently similar blobs popping in and out of existence at sufficient proximity to one another.}
\begin{figure}
%\includegraphics[width=1\textwidth]{\TODO{SKEW}}
\caption{\TODO{}}
\label{fig:skew}
\end{figure}
