\newcommand{\run}[1]{#1-aviI}
\newcommand{\RUN}{\aviI:\;}
\newthought{The results } from the \MI -method are special in that they feature many long-lived eddies (see \cref{fig:histTrackCount-aviI,fig:tracks-age-aviI_shrunk2prepress,fig:defl-age-aviI_shrunk2prepress}),
some of which travelled more than $4000$ km west.
Tracks were recorded throughout the entire world ocean with the only exeptions being an approximately $\deg{20}$-wide stripe along the equator. The highest count of unique eddies is along the Antarctic Circumpolar Current \footnote{abbreviated ACC from here on.} with counts of more than $60$ individual eddy-visits per $\deg{1} \times \deg{1}$-cell. Further eddy-rich regions are the western North-Atlantic throughout the Gulf-Stream and North-Atlantic Current, \textit{Mozambique eddies} \citep{schouten2003eddies} at $\deg{20}$ South along the Mozambique coast, along the Agulhas Current and south of the Cup of Good Hope at $\sim \deg{40}$, along the coasts of Brazil, Chile and all along the Eastern, Southern and Western coasts of Australia (see \cref{fig:MapVisitsBoth-aviI}).

%%....................................F.I.G.U.R.E.............................................
\newthought{Eddies appear and disappear }  throughout the world ocean. For long-lived solid eddies there is a tendency to emerge along western coasts (see \cref{fig:birthsdeathsOneYearAndMore}).

\newthought{The scale } \scale of tracked eddies is similar to that in \citet{them}, yet generally smaller in high latitudes and slightly larger in low latitudes (see \cref{fig:Schelts-aviI}). It is larger than the first-mode baroclinic Rossby Radius by factor of at least $2$ and its meridional profile appears to be seperable into two different regimes; one apparently linear profile in low latitudes and a steeper one equatorwards of $\sim \left| \deg{15} \right|$. Regionall, locations of high meso-scale activity appear to correlate with smaller eddy-scales (see \cref{fig:MapSigma-aviI}).

\newthought{The eastward zonal drift speeds } are slightly slower than the first-mode baroclinic Rossby-Wave phase-speed and agree well with the results from \citet{them}. Propagations is generally west-wards except for regions of sufficiently strong eastward advection as in the ACC and North Atlantic Current (see \cref{fig:velZon-aviI,fig:Schelts-aviI}).

%%....................................F.I.G.U.R.E.............................................
\begin{marginfigure}
		\includegraphics[]{\run{histTrackCount}}
\caption[bla]{\RUN Final age distribtion. x-axis: [days], Left y-axis: [1000]}
\label{\run{fig:histTrackCount}}
\end{marginfigure}
%%....................................F.I.G.U.R.E.............................................
%%....................................F.I.G.U.R.E.............................................
\begin{figure}
	\includegraphics[]{\run{birthsdeathsOneYearAndMore}}
	\caption{\RUN Births are in blue and deaths in green. Size of dots scales to age squared. Only showing tracks older than one year.}
	\label{\run{fig:birthsdeathsOneYearAndMore}}
\end{figure}
%%....................................F.I.G.U.R.E.............................................
%%....................................F.I.G.U.R.E.............................................
%\begin{wrapfigure}{r}{0.6\textwidth}
\begin{marginfigure}
		\includegraphics[width=1\linewidth]{\run{MapVisitsBoth}}
		\caption{\RUN Total count of individual eddies per 1 degree square.}
		\label{\run{fig:MapVisitsBoth}}
\end{marginfigure}
%\end{wrapfigure}
%%....................................F.I.G.U.R.E.............................................
%%....................................F.I.G.U.R.E.............................................
\begin{figure}
		\includegraphics[]{\run{hist-sigmaAt-both}}
		\caption{\RUN TODO}
	\label{\run{fig:hist-sigmaAt-both}}
\end{figure}
%%....................................F.I.G.U.R.E.............................................

%%....................................F.I.G.U.R.E.............................................
\begin{figure*}
		\includegraphics[]{\run{MapSigma}}
		\caption{\RUN TODO}
	\label{\run{fig:MapSigma}}
\end{figure*}
%%....................................F.I.G.U.R.E.............................................

%%....................................F.I.G.U.R.E.............................................
\begin{figure*}
		\includegraphics[]{\run{velZon}}
		\caption{\RUN TODO}
	\label{\run{fig:velZon}}
\end{figure*}
%%....................................F.I.G.U.R.E.............................................
