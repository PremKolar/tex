\section{Detection Methods} \label{subsec:detectmethods}
\begin{itemize}
	\item
	One way to find an eddy in \SSH~is to simply scan for closed contours at different vertical levels and then subject each found ring to a series of geometrical tests to decide whether that contour qualifies. Only if all criteria are met is an eddy found. This method was first used by \citet{Chelton2011} and is a relatively simple yet very effective method. Therefore, as a starting point, this method will be adopted and should also serve as a general definition of what will be referred to as an \textit{eddy} hereafter.
	
\newthought{\citeauthor{Chelton2011}}~set the following threshold criteria for their algorithm:
	\begin{enumerate}
		\item
		The \SSH~values of all of the pixels are above (below) a given \SSH~threshold for anticyclonic (cyclonic) eddies.
		\item
		There are at least \textit{[threshold]} pixels and fewer than \textit{[threshold]} pixels comprising the connected region.
		\item
		There is at least one local maximum (minimum) of \SSH~for anticyclonic (cyclonic) eddies.
		\item
		The amplitude of the eddy is at least \textit{[threshold]}.
		\item
		The distance between any pair of points within the connected region must be less than \textit{[threshold]}.
	\end{enumerate}

\item
Another frequently used method to define an eddy makes use of the 2d deformation tensor $\grad \vec{u}$.
\begin{align}
det(\lambda\vec{I}- \vec{\grad \vec{u}}) =0 \label{eq:OWa}
\end{align}
The sign of its squared eigenvalues indicates whether the flow-field has parabolic, vorticity dominated character, or whether deformation dominates, giving hyperbolic character. Expanding \eqref{eq:OWa} yields
\begin{align}
%\left( \lambda - u_{x} \right) \left( \lambda - v_{y} \right) - u_{y} v_{x}
%&=
%0 \nonumber\\
%%
%\lambda^{2} -\lambda v_{y} -\lambda u_{x} + u_{x} v_{y} - u_{y} v_{x}
%&=
%0 \nonumber \\
%%%
\lambda^{2}
-\lambda
\left( v_{y} + u_{x} \right)
+ u_{x} v_{y}
- u_{y} v_{x}
&=
0 \label{eq:OWb}
\end{align}
Assuming horizontal velocities to be much larger than vertical \ie applying the small aspect-ratio assumption \citep{olbers2012ocean}, the motion becomes 2-dimensional and the continuity equation reduces to $u_{x} = -v_{y}$. Hence 
\begin{align}
%\lambda^{2}
%+ u_{x} v_{y}
%- u_{y} v_{x}
%&=
%0 \nonumber \\
%\lambda^{2}
%- u_{x}^{2}
%- u_{y} v_{x}
%&=
%0 \nonumber \\
\lambda^{2}
&=
\okubo/4
=
 u_{x}^{2}
 +u_{y} v_{x} \label{eq:OWc}
\end{align}
	This is called the Okubo-Weiss-Parameter\footnote{see also \cref{der:okubo}.} $\okubo$ \citep{Okubo1970}.
Its meaning is further elucidated by interpreting \eqref{eq:OWc} as\footnote{see \cref{der:enstro}.}:
\begin{align}
\okubo	
	&=
	s_{n}^{2}
	+
	s_{s}^{2}
	-
	\omega^{2}
+
\div{\vec{u}}
	\\
	&=
	\left( u_{x} - v_{y} \right)^{2}
	+
	\left( v_{x} + u_{y} \right)^{2}
	-
	\left( v_{x} - u_{y} \right)^{2}
	+
	\left( u_{x} + v_{y} \right)^{2}
	\nonumber \\
	&=
4 u_{x}^{2} 
	+
 4v_{x}u_{y} \nonumber
\end{align}
where $s_{n/s}$ are the normal respective shear components of strain. Its sign thus describes the field's tendency for either vorticity- or shear-dominated motion \citep{Isern-Fontanet2006}.
	 An area of large negative values of $\okubo$ indicates high enstrophy density compared to gradients of kinetic energy \citep{Weiss1991}, thus indicating little friction paired with high momentum \ie a vorticity-dominated field as would be found in a coherent, angular-momentum-conserving entity. 
	  Positive values on the other hand indicate motion dominated by deformation as \eg in-between two vortices of opposite sign. The fourth term, which is irrelevant here, represents divergence and can here be interpreted as negative \textit{vortex stretching} (\eg bathtub sink). 
	 
	 \newthought{As } useful as this parameter seems, it turns out that using it to identify eddies is often not practical.
	\Citet{Chelton2011} name 3 major drawbacks:
	\begin{itemize}
		\item
		\textit{ No single threshold value for $\okubo$ is optimal for the entire World Ocean. Setting the threshold too high can result in failure to identify small eddies, while a threshold that is too low can lead to a definition of eddies with unrealistically large areas that may encompass multiple vortices, sometimes with opposite polarities. }
		\item
		$\okubo$ is highly susceptible to noise in the \SSH~field. Especially when velocities are calculated from geostrophy, the sea surface has effectively
		been differentiated twice and then squared, exacerbating small incontinuities in the data.
		\item
		\textit{The third problem with the W-based method is that the interiors of eddies defined by closed contours of W do not generally coincide with closed contours of SSH. The misregistration of the two fields is often quite substantial. }
	\end{itemize}
	In summary, the $\okubo$-method critically hinges on the necessary assumption of a smooth, purely geostrophic \SSH~topography and is therefor inferior to the approach of scanning for closed SSH-contours directly (as was done so by \citeauthor{Chelton2011}) (see also \citet{Zhang2013}).

\end{itemize}
\newpage
