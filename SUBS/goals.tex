
\label{sec:goals}



Not all of these will be achieved of course. They will be worked through in the order they appear until the time scope of this thesis is over.
{\bfseries
\begin{enumerate}

\item
Implement an efficient, parallelized, portable, modular algorithm entirely in matlab with focus on straightforwardness, easy maintainability and tweakability.

\item
Mimic the algorithm as described by \cite{Chelton2011} and apply it to the model data.

\item
Repeat all of the analysis done by \cite{Chelton2011} and compare.

\item
Test alternative definition of the \textit{center} of an eddy as the absolute value of the center of volume of the eddy measured from reference height ($z$ of its perimeter).

\item
Test alternative shape-threshold criterion based on isoperimetric coefficient \ie the eddy's sphericity. \footnote{\cite{Chelton2011} guaranteed sphericity by demanding that \textit{The distance between any pair of points within the connected region must be less than a specified maximum.}}.

\item
Compare theoretical phase and group velocities of Rossby waves at wavelength set to determined scale of eddies with eddy velocities.

\item
Compare theoretical wave length of Rossby waves at phase/group speed set to determined drift speed of eddies with eddy scale, similar to \citep{Tulloch2009}.

\item
Compare scales to those derived from linear stability analysis by \eg \cite{Vollmer2013a,eden2012implementing,Smith2009,griesel2013eulerian}.
\end{enumerate}
}

\begin{enumerate}
\setcounter{enumi}{8}
\item
Analyze eddy-genesis and fate. Look at mean values \footnote{either means in time or means in space.} $\Ro$, $\Bu$, $\Rh$, track, scale, as a function of location.

\item
Down-sample the model data, and compare runs for model/satellite at identical
resolution and code.

\item
Go sub-surface and apply the algorithm to at least one other depth than the surface, with the hypothesis that this might help to avoid meso-scale noise in highly turbulent locations.

\item
Use findings from model to improve Aviso tracking.
\item
Filter the SSH-data via Galilean-/LES-decomposition \cite{Adrian2000a} or simply
by a filter created from determined mean shapes of eddies as a function of
location. The motivation is again to successfully track eddies through zones of
strong turbulence.

\item
focus on regions of large Rossby number and regions of strong turbulence (ACC).

\item
Calculate the \emph{effective} Rossby radius as mentioned in \cite{Vollmer2013a} and compare to eddy scales.

\item
Include short lived eddies. \cite{Chelton2011} only analyzed eddies with life times longer than 16 weeks in order to exclude less \textit{robust} eddies for which they did not fully trust their algorithm to be capable of tracking correctly.

\item
Test whether tracers inside eddy stay on isopycnals.

\item
Determine tracks of tracers inside eddy and compare to the eddie's track.


\end{enumerate}
