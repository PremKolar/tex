% ######################################################################
% OTHER
% ######################################################################
\newcommand{\rnm}[1]{\romannumeral #1}
\newcommand{\Rmn}[1]{\Roman #1}
\newcommand{\footder}[1]{\footnote{See section \cref{#1} for derivation.}}
\definecolor{colboxcolor}{HTML}{9ECBFF}
\definecolor{cboxcolor}{HTML}{EEEEEE}
\definecolor{speedboxcolor}{HTML}{F7F7F7}
\definecolor{filtercolor}{HTML}{FFFFFF}
\definecolor{legendcol}{HTML}{7BBCBD}
\newcommand{\HRule}{\rule{\linewidth}{0.5mm}} % Defines a new command for the horizontal lines, change thickness here
\newcommand{\ie}{\textit{i.\hairsp{}e.}\xspace} % Command to print i.e.
\newcommand{\eg}{\textit{e.\hairsp{}g.}\xspace} % Command to print e.g.
\newcommand{\Eg}{E.g.\xspace}
\newcommand{\Ie}{I.e.\xspace}
\newcommand{\etc}{\@ifnextchar{.}{etc}{etc.\@\xspace}}
\newcommand{\litem}[2]{\def\@itemlabel{\textbf{#1}}\item\def\@currentlabel{\textbf{#1}}\label{#2}}
\newcommand{\derref}[1]{\footnote{see\fcolorbox{white}{gray!20}{Derivation \cref{#1}}}}
\newcommand{\derrefs}[1]{\footnote{see\fcolorbox{white}{gray!20}{Derivations \cref{#1}}}}


%  #####################################################################
% MATH
% ######################################################################

\let\oldeqref\eqref
\renewcommand{\vec}[1]{\bm{#1}}
\let\oldhat\hat
\newcommand{\ten}[1]{\mathbb{\bm{#1}}}
\newcommand{\tent}[1]{\mathbb{\bm{#1}}^{\mathsmaller T}}
\renewcommand{\l}{\left(}
 \renewcommand{\r}{\right)}
\newcommand{\pr}{\partial}
\newcommand{\grad}{\vec{\nabla}}
\newcommand{\arccosh}{\mathrm{arccosh}}
\newcommand{\lapl}{\vec{\triangle}}
\newcommand{\curl}{\grad \times}
\newcommand{\gradt}{ \breve{\grad}}
\renewcommand{\div}{\grad \cdot}
\newcommand{\norm}[1]{\left\lVert#1\right\rVert}
\renewcommand{\eqref}[1]{equation \oldeqref{#1}}
\newcommand{\eqsref}[1]{equations \oldeqref{#1}}
\newcommand{\Eqref}[1]{Equation \oldeqref{#1}}
\newcommand{\Eqsref}[1]{Equations \oldeqref{#1}}
\newcommand{\dpr}[2]{\frac{\partial#1}{\partial#2}}
\newcommand{\Dpr}[2]{\frac{D#1}{D#2}}
\newcommand{\Dprs}[2]{\frac{D^{\star}#1}{D#2}}
\newcommand{\advec}[1]{\vec{u} \cdot \vec{\nabla} #1}
\newcommand{\Unit}[1]{\left[#1\right]}
\newcommand{\unitvec}[1]{\oldhat{\vec{#1}}}
\newcommand{\tvec}[1]{\underaccent{\neg}{\vec{#1}}}
\newcommand{\tsca}[1]{\underaccent{\neg}{#1}}
\newcommand{\order}[1]{\mathcal{O}\left( 10^{#1}\right)}
\newcommand{\sign}[1]{\mathrm{sgn}\left(#1\right)}
\newcommand{\abs}[1]{\left|#1\right|}
\DeclareMathOperator{\tr}{Tr}
\newcommand{\eexp}[1]{ \mathrm{e}^{#1}}
\renewcommand{\i}{\mathrm{i}}
\newcommand{\Cdot}[2]{\vec{#1} \cdot \vec{#2}}

% ######################################################################
% COUNTERS
% ######################################################################

\newcounter{cbox}
\newcounter{filter}

% ######################################################################
% ENVIRONTMENTS
% ######################################################################

\newenvironment{cbox}[1]
{
\refstepcounter{cbox}\par\medskip\noindent
\def\FrameCommand{\colorbox{cboxcolor}}%
\MakeFramed{\advance\hsize-\width \FrameRestore\cornersize{0.9}}
\begin{scriptsize}
\begin{center}
\textbf{Box~\thecbox. #1}
\end{center}
}
{
\end{scriptsize}
\endMakeFramed
}

\newenvironment{speedbox}[1]
{
\refstepcounter{cbox}\par\medskip\noindent
\def\FrameCommand{\colorbox{speedboxcolor}}%
%\MakeFramed{\advance\hsize-\width \FrameRestore\cornersize{0.9}}
%\MakeFramed{\FrameRestore\cornersize{0.99}}
\begin{scriptsize}
\begin{center}
\textbf{Box~\thecbox. #1}
\end{center}
}
{
\end{scriptsize}
%\endMakeFramed
}

\newenvironment{filter}[2]
{
\refstepcounter{filter}\par\medskip\noindent
\begin{large}
\textit{Contour filter~\thefilter} \textbf{#1}\\
\noindent \mcode{function #2}\\
\end{large}
\noindent
}
{
}

\newenvironment{colbox}[1]
{
	\def\FrameCommand{\colorbox{colboxcolor}}%
	\MakeFramed{\advance\hsize-\width \FrameRestore\cornersize{0.9}}
	\begin{scriptsize}
	\section{#1}
}
{
    \end{scriptsize}
	\endMakeFramed
}

\newenvironment{Declaration}{   
  \chapter*{Declaration of Authorship}
  \addcontentsline{toc}{chapter}{Declaration of Authorship}  
}
{
\clearpage
}

\newenvironment{acknol}{   
  \chapter*{Acknoledgements}
  \setstretch{1.3}
  %\addcontentsline{toc}{chapter}{Acknoledgements}  
}
{
\clearpage
}

%  #####################################################################
% STYLES
% ######################################################################

\mdfdefinestyle{definition}
{linewidth=1,
backgroundcolor=yellow!40,
outerlinecolor=blue!70!black,
frametitlebackgroundcolor=gray!20,
% frametitlerule=true,
innertopmargin=\topskip,}


\mdfdefinestyle{derivation}
{linewidth=1,
backgroundcolor=yellow!40,
outerlinecolor=blue!70!black,
frametitlebackgroundcolor=gray!10,
% frametitlerule=true,
innertopmargin=\topskip,}

\mdfdefinestyle{codepiece}
{linewidth=15pt,
linecolor=black,
%frametitlerule=true,
backgroundcolor=red!1,
outerlinecolor=black,
frametitlebackgroundcolor=blue!10,
innertopmargin=\topskip,}

\mdfdefinestyle{function}
{%linewidth=15pt,
%linecolor=black,
%frametitlerule=true,
%backgroundcolor=red!1,
%outerlinecolor=black,
%frametitlebackgroundcolor=blue!10,
innertopmargin=-.5cm,}

\mdfdefinestyle{derivation}
{linecolor=red,
outerlinewidth=2,
leftmargin=40mm,
rightmargin=40mm,
linewidth=2,
frametitlerule=true,
frametitlebackgroundcolor=gray!20,
innertopmargin=\topskip,roundcorner=10pt,}

\mdfdefinestyle{eddy}
{linewidth=.5,
% align=center,
% backgroundcolor=yellow!40,
outerlinecolor=black,
frametitlebackgroundcolor=gray!20,
% frametitlerule=true,
innertopmargin=\topskip,}

%  #####################################################################
% mdTHEOREMS
% ######################################################################

\mdtheorem[style=definition]{definition}{Definition}

\mdtheorem[style=codepiece]{codepiece}{Code}[chapter]

\mdtheorem[style=function]{function}{sub-routine}[section]

\mdtheorem[style=derivation]{derivation}{Derivation}

\mdtheorem[style=eddy]{eddy}{Vortex}[chapter]

\mdtheorem[style=eddy]{turbu}{Turbulence}[chapter]


%  #####################################################################
%% legend stuff
% ######################################################################

\newcommand{\Bu}[0]{\mathrm{\hyperref[def:Bu]{Bu}}\;}
\newcommand{\Ro}[0]{\mathrm{\hyperref[def:Ro]{Ro}}\;}
\renewcommand{\Re}[0]{\mathrm{\hyperref[def:Re]{Re}}\;}
\newcommand{\Rh}[0]{\mathrm{\hyperref[def:Rh]{R_{\beta}}}\;}
\newcommand{\Lr}[0]{\mathrm{\hyperref[def:Lr]{L_{R}}}\;}
\newcommand{\Lb}[0]{\mathrm{\hyperref[def:Lb]{L_{\beta}}}\;}
\newcommand{\h}[0]{\hyperref[def:h]{h}}
\newcommand{\B}[0]{\hyperref[def:B]{ \vec{B}}}
\newcommand{\Ek}[0]{\hyperref[def:Ek]{E_k}}
\newcommand{\Em}[0]{\hyperref[def:Em]{E_m}}
\newcommand{\scale}[0]{\hyperref[def:scale]{$\sigma$}\;}
\newcommand{\CoV}[0]{\hyperref[filter:CoV]{CoV}\;}
\newcommand{\enstro}[0]{\hyperref[def:enstro]{\varepsilon}}
\newcommand{\f}[0]{\mathit{\hyperref[def:f]{f}}\;}
\newcommand{\dfdy}[0]{\mathrm{\hyperref[def:beta]{\beta}}\;}
\newcommand{\g}[0]{\mathit{\hyperref[def:g]{g}}\;}
\newcommand{\okubo}[0]{\mathrm{\hyperref[def:okubo]{O_w}}\;}
\newcommand{\SSH}[0]{\hyperref[def:SSH]{SSH}\;}
\newcommand{\IQ}[0]{\mathrm{\hyperref[def:IQ]{IQ}}\;}
\newcommand{\rG}[0]{\hyperref[def:rG]{\mathfrak{r}}\;}
\newcommand{\aviI      }[0]{\hyperref[def:aviI ]{aviso-MI                      }\@\xspace}
\newcommand{\aviII     }[0]{\hyperref[def:aviII]{aviso-MII                     }\@\xspace}
\newcommand{\pToaII    }[0]{\hyperref[def:p2aII]{pop2avi-MII                   }\@\xspace}
\newcommand{\popSevenII}[0]{\hyperref[def:pop7II]{POP-7day-MII                 }\@\xspace}
\newcommand{\popOneIISO}[0]{\hyperref[def:pop1IISO]{POP-1day-MII-Southern-Ocean}\@\xspace}
\newcommand{\MI}[0]{\hyperref[box:MI]{MI}\@\xspace}
\newcommand{\MII}[0]{\hyperref[box:MII]{MII}\@\xspace}
\newcommand{\POP}[0]{\hyperref[def:POP]{POP}\@\xspace}
\newcommand{\AVISO}[0]{\hyperref[def:AVISO]{AVISO}\@\xspace}
\newcommand{\TODO}[1]{\textbf{\textcolor{red}{TODO:#1}}}


\newcommand{\Enstro}{\mathcal{E}}
\newcommand{\inta}[1]{ \int_A #1 \; \mathrm{d}A}
\newcommand{\intm}[1]{ \frac{1}{A} \int_A #1 \; \mathrm{d}A}
\newcommand{\dint}[1]{ \; \mathrm{d}#1}
%\newcommand{\d}[1]{\mathrm{d}#1}
\newcommand{\expp}[1]{\mathrm{e}^{#1}}
\newcommand{\INT}[4]{\int_{#1}^{#2} #3 \; \mathrm{d}#4}
\newcommand{\intms}[1]{ \left< #1 \right>}
%\renewcommand{\deg}[1]{#1^\circ}
\renewcommand{\deg}[1]{\ang{#1}}
\newcommand{\km}[1]{#1 \mathrm{km}}
\newcommand{\decom}[1]{\overline{#1} + #1'}
\newcommand{\inbr}[1]{\left( #1 \right)}
\newcommand{\ol}[1]{\overline{#1}}
\newcommand{\timesES}[2]{\epsilon_{jki} #1_j #2_k \unitvec{e}_i}
\newcommand{\oh}[0]{\frac{1}{2}}
% ######################################################################
% VARIABLES
% ######################################################################
\newcommand*{\supervisor}[1]{\def\supname{#1}}
\newcommand*{\thesistitle}[1]{\def\ttitle{#1}}
\newcommand*{\examiner}[1]{\def\examname{#1}}
\newcommand*{\degree}[1]{\def\degreename{#1}}
\newcommand*{\authors}[1]{\def\authornames{#1}}
\newcommand*{\addresses}[1]{\def\addressnames{#1}}
\newcommand*{\university}[1]{\def\univname{#1}}
\newcommand*{\UNIVERSITY}[1]{\def\UNIVNAME{#1}}
\newcommand*{\department}[1]{\def\deptname{#1}}
\newcommand*{\DEPARTMENT}[1]{\def\DEPTNAME{#1}}
\newcommand*{\group}[1]{\def\groupname{#1}}
\newcommand*{\GROUP}[1]{\def\GROUPNAME{#1}}
\newcommand*{\faculty}[1]{\def\facname{#1}}
\newcommand*{\FACULTY}[1]{\def\FACNAME{#1}}
\newcommand*{\subject}[1]{\def\subjectname{#1}}
\newcommand*{\keywords}[1]{\def\keywordnames{#1}}
\newcommand\btypeout[1]{\bhrule\typeout{\space #1}\bhrule}
\newcommand\bhrule{\typeout{------------------------------------------------------------------------------}}
% ######################################################################

\newcommand{\hangp}[1]{\makebox[0pt][r]{(}#1\makebox[0pt][l]{)}} % New command to create parentheses around text in tables which take up no horizontal space - this improves column spacing
\newcommand{\hangstar}{\makebox[0pt][l]{*}} % New command to create asterisks in tables which take up no horizontal space - this improves column spacing
\newcommand{\monthyear}{\ifcase\month\or January\or February\or March\or April\or May\or June\or July\or August\or September\or October\or November\or December\fi\space\number\year} % A command to print the current month and year
\newcommand{\openepigraph}[2]{ % This block sets up a command for printing an epigraph with 2 arguments - the quote and the author
\begin{fullwidth}
\sffamily\large
\begin{doublespace}
\noindent\allcaps{#1}\\ % The quote
\noindent\allcaps{#2} % The author
\end{doublespace}
\end{fullwidth}
}
\newcommand{\blankpage}{\newpage\hbox{}\thispagestyle{empty}\newpage} % Command to insert a blank page
\newcommand{\hlred}[1]{\textcolor{Maroon}{#1}} % Print text in maroon
\newcommand{\hangleft}[1]{\makebox[0pt][r]{#1}} % Used for printing commands in the index, moves the slash left so the command name aligns with the rest of the text in the index
\newcommand{\hairsp}{\hspace{1pt}} % Command to print a very short space
\newcommand{\na}{\quad--} % Used in tables for N/A cells
\newcommand{\measure}[3]{#1/#2$\times$\unit[#3]{pc}} % Typesets the font size, leading, and measure in the form of: 10/12x26 pc.
\newcommand{\tuftebs}{\symbol{'134}} % Command to print a backslash in tt type in OT1/T1
\providecommand{\XeLaTeX}{X\lower.5ex\hbox{\kern-0.15em\creflectbox{E}}\kern-0.1em\LaTeX}
\newcommand{\tXeLaTeX}{\XeLaTeX\index{XeLaTeX@\protect\XeLaTeX}} % Command to print the XeLaTeX logo while simultaneously adding the position to the index
\newcommand{\doccmdnoindex}[2][]{\texttt{\tuftebs#2}} % Command to print a command in texttt with a backslash of tt type without inserting the command into the index
\newcommand{\doccmddef}[2][]{\hlred{\texttt{\tuftebs#2}}\label{cmd:#2}\ifthenelse{\isempty{#1}} % Command to define a command in red and add it to the index
{ % If no package is specified, add the command to the index
\index{#2 command@\protect\hangleft{\texttt{\tuftebs}}\texttt{#2}}% Command name
}
{ % If a package is also specified as a second argument, add the command and package to the index
\index{#2 command@\protect\hangleft{\texttt{\tuftebs}}\texttt{#2} (\texttt{#1} package)}% Command name
\index{#1 package@\texttt{#1} package}\index{packages!#1@\texttt{#1}}% Package name
}}
\newcommand{\doccmd}[2][]{% Command to define a command and add it to the index
\texttt{\tuftebs#2}%
\ifthenelse{\isempty{#1}}% If no package is specified, add the command to the index
{%
\index{#2 command@\protect\hangleft{\texttt{\tuftebs}}\texttt{#2}}% Command name
}
{%
\index{#2 command@\protect\hangleft{\texttt{\tuftebs}}\texttt{#2} (\texttt{#1} package)}% Command name
\index{#1 package@\texttt{#1} package}\index{packages!#1@\texttt{#1}}% Package name
}}


% A bunch of new commands to print commands, arguments, environments, classes, etc within the text using the correct formatting
\newcommand{\docopt}[1]{\ensuremath{\langle}\textrm{\textit{#1}}\ensuremath{\rangle}}
\newcommand{\docarg}[1]{\textrm{\textit{#1}}}
\newcommand{\docenv}[1]{\texttt{#1}\index{#1 environment@\texttt{#1} environment}\index{environments!#1@\texttt{#1}}}
\newcommand{\docenvdef}[1]{\hlred{\texttt{#1}}\label{env:#1}\index{#1 environment@\texttt{#1} environment}\index{environments!#1@\texttt{#1}}}
\newcommand{\docpkg}[1]{\texttt{#1}\index{#1 package@\texttt{#1} package}\index{packages!#1@\texttt{#1}}}
\newcommand{\doccls}[1]{\texttt{#1}}
\newcommand{\docclsopt}[1]{\texttt{#1}\index{#1 class option@\texttt{#1} class option}\index{class options!#1@\texttt{#1}}}
\newcommand{\docclsoptdef}[1]{\hlred{\texttt{#1}}\label{clsopt:#1}\index{#1 class option@\texttt{#1} class option}\index{class options!#1@\texttt{#1}}}
\newcommand{\docmsg}[2]{\bigskip\begin{fullwidth}\noindent\ttfamily#1\end{fullwidth}\medskip\par\noindent#2}
\newcommand{\docfilehook}[2]{\texttt{#1}\index{file hooks!#2}\index{#1@\texttt{#1}}}
\newcommand{\doccounter}[1]{\texttt{#1}\index{#1 counter@\texttt{#1} counter}}
\newcommand{\vdqi}{\textit{VDQI}\xspace}
\newcommand{\ei}{\textit{EI}\xspace}
\newcommand{\ve}{\textit{VE}\xspace}
\newcommand{\be}{\textit{BE}\xspace}
\newcommand{\VDQI}{\textit{The Visual Display of Quantitative Information}\xspace}
\newcommand{\EI}{\textit{Envisioning Information}\xspace}
\newcommand{\VE}{\textit{Visual Explanations}\xspace}
\newcommand{\BE}{\textit{Beautiful Evidence}\xspace}
\newcommand{\TL}{Tufte-\LaTeX\xspace}

%----------------------------------------------------------------------------------------
\newenvironment{docspec}{\begin{quotation}\ttfamily\parskip0pt\parindent0pt\ignorespaces}{\end{quotation}}
