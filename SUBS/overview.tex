\newthought{Due}~to the inherently technical character of the matter, large parts of this text are dedicated to details of the algorithm (see \cref{chap:algorithm}). Oceanographic results are treated in the \href{chap:results}{Results}- and \href{chap:discussion}{Discussion}-chapters. The \href{chap:THEORY}{Methods and Theory} chapter outlines the different approaches to detect eddies from \SSH, explains the mechanism of the \href{subsec:speeds}{westward drift} of eddies, discusses the \href{subsec:horScales}{horizontal scale} of eddies and explains its relevance to climate models and shows the \href{sec:satvsmod}{differences} of the two data sets (satellite vs model).
The \href{chap:history}{History} chapter introduces a handful of relevant historical papers.
Results of the different censuses are presented in the \href{chap:results}{Results} chapter and discussed in the \href{chap:discussion}{Discussion}. The quintessence of the findings is summarized in the \href{chap:conclusion}{Conclusions}-chapter. Several further, thus far incomplete, topics are touched in the \href{appendix:futureTopics}{Ideas for the Future}-appendix. The theoretical \href{appendix:eddy_cat}{Eddy}- and \href{appendix:eddy_cat}{Turbulence}-appendices and the \href{appendix:derivations}{Derivations}-appendix reflect preliminary theoretical thoughts of the author and are meant as a bonus.


%Since focus is on horizontal scales, drift speeds and the comparison of results between the \AVI~product and \SSH-data from the \POP~ocean model, sections generally focus on either of these three topics. 
