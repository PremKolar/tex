\renewcommand{\run}[1]{#1-pop7II}
\renewcommand{\RUN}{pop7-\MII: }

\newthought{The model data } delivers slightly more total tracks with a similar 2-fold dominance of cyclones over anti-cyclones (compare \cref{fig:histTrackCount-aviII,fig:histTrackCount-pop7II}). Similar to \aviII, very long tracks are fewer than via \aviI \footnote{\aviI~features $3000$ tracks that are older than $400$ days, while both \MII~methods have only $\sim1000$ of such.}. The regional pattern looks somewhat similar to the satellite patterns in terms of which regions feature the strongest eddy activity. With the exception of an unrealistic abundance of eddies right along the Antarctic coast  where no eddies were detected for the satellite data likely due to sea ice and/or the inherent lack of polar data due to the satellites' orbit-inclinations.
%%....................................F.I.G.U.R.E.............................................
\begin{marginfigure}
		\includegraphics[]{\run{histTrackCount}}
\caption[bla]{\RUN Final age distribution. x-axis: [days], Left y-axis: [1000]}
\label{\run{fig:histTrackCount}}
\end{marginfigure}
%%....................................F.I.G.U.R.E.............................................

\newthought{The } more important difference between model- and satellite regional distributions is that the model results indicate significantly less eddy activity away from regions of strong \SSH~gradients, in the open ocean away from coasts and strong currents. The algorithm also detects hardly any eddy tracks in tropical regions (see \cref{fig:MapVisitsBoth-aviI}). This regional heterogeneity in eddy-activity in the model data is also reflected in the distribution of eddy amplitudes (see \cref{fig:TrackPeakampto_ellipseAntiCycsCrpd}).

\newthought{The scale $\scale$ } is generally smaller for the model-data-based analysis than for any satellite-based analyses, especially so in high latitudes.

\newthought{Westward drift speeds } look regionally similar to those from satellite data (\cref{fig:MapSigmaVelZon-pop7II,fig:velZon-aviI}). In the zonal mean their magnitude is below those from satellite (see \cref{fig:ScheltsAll}).
%%....................................F.I.G.U.R.E.............................................
%~ %\begin{figure}
	%~ %\includegraphics[]{\run{birthsdeathsOneYearAndMore}}
	%~ %\caption{\RUN Births are in blue and deaths in green. Size of dots scales to age squared. Only showing tracks older than one year.}
	%~ %\label{\run{fig:birthsdeathsOneYearAndMore}}
%~ %\end{figure}
%%....................................F.I.G.U.R.E.............................................
%%....................................F.I.G.U.R.E.............................................
%\begin{wrapfigure}{r}{0.6\textwidth}
\begin{marginfigure}
		\includegraphics[width=1\linewidth]{\run{MapVisitsBoth}}
		\caption{\RUN Total count of individual eddies per 1 degree square.}
		\label{\run{fig:MapVisitsBoth}}
\end{marginfigure}
%\end{wrapfigure}
%%%....................................F.I.G.U.R.E.............................................
%\begin{figure}
		%\includegraphics[]{\run{MapSigma}}
		%\caption{\RUN \capS}
	%\label{\run{fig:MapSigma}}
%\end{figure}
%%%....................................F.I.G.U.R.E.............................................

%%%....................................F.I.G.U.R.E.............................................
%\begin{figure}
		%\includegraphics[]{\run{velZon}}
		%\caption{\RUN \capU}
	%\label{\run{fig:velZon}}
%\end{figure}
%%%....................................F.I.G.U.R.E.............................................

%%....................................F.I.G.U.R.E.............................................
\begin{figure*}
		\includegraphics[]{\run{MapSigmaVelZon}}
		\caption{\RUN. Top: \capS. Bottom: \capU.}
	\label{\run{fig:MapSigmaVelZon}}
\end{figure*}
%%....................................F.I.G.U.R.E.............................................





%%%....................................F.I.G.U.R.E.............................................
%\begin{figure}
		%\includegraphics[]{\run{hist-sigmaAt-both}}
		%\caption{\RUN TODO}
		%\label{\run{fig:hist-sigmaAt-both}}
%\end{figure}
%%%....................................F.I.G.U.R.E.............................................
