\renewcommand{\run}[1]{#1-aviII}
\renewcommand{\RUN}{\AVI~-\MII:\;}


%%....................................F.I.G.U.R.E.............................................
\begin{figure}
\includegraphics[]{tracks-age-aviII_shrunk2prepress}
\caption{\MII: (see \cref{fig:tracks-age-aviI_shrunk2prepress})}
\label{fig:tracks-age-aviII_shrunk2prepress}
\end{figure}
%%....................................F.I.G.U.R.E.............................................


\newthought{The}~$\IQ$-based method results in approximately the same total amount of tracks as the \MI -method used in \cref{section:aviI} (see \cref{fig:histTrackCount-aviI,fig:histTrackCount-aviII}). The difference is that tracks here are generally much shorter, meaning that less eddies are detected at any given point in time.
The scale $\scale$ is now smaller than that from \citet{Chelton2011} for all latitudes in zonal- mean as well as median. Westward drift speeds are almost identical to those in \cref{section:aviI}.

%%....................................F.I.G.U.R.E.............................................
\begin{marginfigure}
		\includegraphics[]{\run{histTrackCount}}
\caption[bla]{\RUN Final age distribtion. x-axis: [days], Left y-axis: [1000]}
\label{\run{fig:histTrackCount}}
\end{marginfigure}
%%....................................F.I.G.U.R.E.............................................
%%....................................F.I.G.U.R.E.............................................
\begin{figure}
	\includegraphics[]{\run{birthsdeathsOneYearAndMore}}
	\caption{\RUN Births are in blue and deaths in green. Size of dots scales to age squared. Only showing tracks older than one year.}
	\label{\run{fig:birthsdeathsOneYearAndMore}}
\end{figure}
%%....................................F.I.G.U.R.E.............................................
%%....................................F.I.G.U.R.E.............................................
%\begin{wrapfigure}{r}{0.6\textwidth}
\begin{marginfigure}
		\includegraphics[width=1\linewidth]{\run{MapVisitsBoth}}
		\caption{\RUN Total count of individual eddies per 1 degree square.}
		\label{\run{fig:MapVisitsBoth}}
\end{marginfigure}
%\end{wrapfigure}
%%%....................................F.I.G.U.R.E.............................................
%\begin{figure}
		%\includegraphics[]{\run{Schelts}}
		%\caption{\RUN Left: Zonal-mean drift speed (cyan) fit to Fig 22 of \protect{\citep{Chelton2011}} (Background) . Right: $\sigma$ and $L_{e}$ fit to Fig. 12 of their paper.}
	%\label{\run{fig:fSchelts}}
%\end{figure}
%%%....................................F.I.G.U.R.E.............................................

%%....................................F.I.G.U.R.E.............................................
\begin{figure*}
		\includegraphics[]{\run{MapSigma}}
		\caption{\RUN \capS}
	\label{\run{fig:MapSigma}}
\end{figure*}
%%....................................F.I.G.U.R.E.............................................

%%....................................F.I.G.U.R.E.............................................
\begin{figure*}
		\includegraphics[]{\run{velZon}}
		\caption{\RUN \capU}
	\label{\run{fig:velZon}}
\end{figure*}
%%....................................F.I.G.U.R.E.............................................
