Intuitively any translative motion of a vortex should stem from an asymmetry of forces as in an imperfectly balanced gyroscope wobbling around and translating across a table.
The main effects that cause a quasi-geostrophic ocean eddy to translate laterally can rel. easily be explained heuristically.


\newpage
\begin{speedbox}{Lateral Density Gradient}
\label{box:speed_dens}
Consider a mean layer-thickness gradient $\dpr{h}{x}>0$ somewhere in the high northern latitudes and a geostrophic, positive density anomaly within that layer.
In other words, a high-pressure vortex or an anti-cyclonic eddy with length scale $L\approx \mathrm{L_{R}}$. Hence a vorticity budget dominated by advection of relative vorticity and vortex stretching. \\
Next consider a parcel of water within the eddy's western flank.
As the clockwise rotating eddy advects the parcel east via its northern side the water-column comprising said fluid will be stretched vertically as it is advected towards larger depths. In order to maintain total vorticity a small new relative-vorticity term is introduced.
Since the vorticity budget is dominated by the planetary component, this new term has sign of $\f$ \ie positive. The net effect is that the absolute value of the parcel's (negative) relative vorticity is reduced \ie the circular flow around the eddy's center is slowed (via term $C$ in \eqref{eq:vort7}), leading to an accumulation of water on the eddy's northern flank. \\
The opposite effect holds for a parcel advected along the southern side from east to west. Then, total (positive) vorticity is reduced so that the already negative relative vorticity becomes even more \textit{negative}, resulting in an accelerated flow south to the eddy's core.
To conserve volume the accumulation north and the decumulation of water south, the eddy is slowly pushed south.\\
Note  that the rotational sense of the eddy is irrelevant here. In the cyclonic case, even though columns are stretched north and south directly opposite to the anti-cyclonic case, the effect on absolute values of relative vorticity is directly opposite too.
\Ie vortex stretching, even though now in the south, now increases the flow, as the added term of rel. vorticity now is of the same sign as the eddy itself. \\
The drift direction  is hence dictated by the sign of $\f$, so that eddies in the northern hemisphere will be pushed along gradients with the shallower water always on their right and vice versa on the southern hemisphere.
\end{speedbox}
%\newpage

\begin{speedbox}{\textit{Planetary Lift}}
\label{box:speed_planlift}
Assume now that $\beta L$ be comparable or larger even than $f_{0}-\omega$ from the previous example.
Then, all fluid adjacent to the eddy on its northern and southern flanks will be transported meridionally, thereby be tilted with respect to $\Omega$ and hence acquire relative vorticity to compensate.
All fluid transported north will balance the increase in planetary vorticity with a decrease in relative vorticity and vice versa for fluid transported south. This is again independent of the eddy's sense and in this case also independent of hemisphere since $\dpr{f}{y}=\beta>0$ for all latitudes.
The result is that small negative vortices to the northern and small positive vortices to the southern flank of eddies will push them west.
\end{speedbox}

\begin{speedbox}{Eddy-Internal $\beta$-Effect}
\label{box:speed_beta}
In the later case clearly particles within the vortex undergo a change in planetary vorticity as well.
Or from a different point of view, since $U \sim \grad p/f  $, and noting that the pressure gradient is the driving force here and hence fix at first approximation, particles drifting north will decelerate and those drifting south will accelerate.
In order to maintain mass continuity, the center of volume will be shifted west for an anti-cyclone and east for a cyclone.
Another way to look at it is to note that the only way for the discrepancy in Coriolis acceleration north and south, whilst maintaining constant eddy-relative particle speed, is to superimpose a zonal drift velocity so that net particle velocities achieve symmetric Coriolis acceleration.
\end{speedbox}

\TODO{equations to follow \citet{Cushman-Roisin1990} \citet{VanLeeuwen2007}}
