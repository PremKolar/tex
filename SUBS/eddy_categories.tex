\label{chap:eddy_cat}
Starting from the considerations for \eqsref{eq:theory6} and introducing a variable
density, the momentum equations and the
$z$-component of the vorticity equation read:
%%--------------------------------------------------------------------
\begin{subequations}
\begin{align}
	\left(\dpr{\vec{u}}{t}\right)^{\rnm{1}}
	+
	\left(\vec{u} \cdot \grad \vec{u}\right)^{\rnm{2}}
	+
	\left(\vec{f}_{0} \times \vec{u}\right)^{\rnm{3}}
	+
	\left(\beta y \times \vec{u}\right)^{\rnm{4}}
	&=
	\left(- g \grad \h \right)^{\rnm{5}}
	\label{eq:NS7} \\
	%%####################################################################
	%%####################################################################
	\div \vec{u}
	&=
	0 \label{eq:konti7}\\
	%%####################################################################
	%%####################################################################
	\left(\Dpr{\omega}{t}\right)^A
	+
	\left(\Dpr{f}{t}\right)^B
	&=
	\left(f \dpr{w}{z} \right)^C
	+
	\left(\omega \dpr{w}{z} \right)^D \label{eq:vort7}
	%%####################################################################
	%%####################################################################
	%\frac{D \Ek}{D t}
	%&=
	%-
	%\vec{u}_{h} \cdot \frac{1}{\rho} \grad_{h} p
	%+
	%\nu \left(
	%\frac{1}{2}\grad^2 \vec{u}^2 - \norm{\grad \vec{u}}^2
%\right)
	%\label{eq:Ekin1}\\
	%%%####################################################################
	%%%####################################################################
	%\frac{D \Em}{D t}
	%&=
	%\nu	\left(
	%\frac{1}{2}\grad^2 \vec{u}^2 - \norm{\grad \vec{u}}^2
	%\right)
	%\label{eq:Emech1}\\
	%%%####################################################################
	%%%####################################################################
	%\frac{D \enstro}{D t}
	%&=
	%\vec{\omega}\cdot \left( \vec{\vec{\omega}_{a}} \cdot \grad \right) \vec{\vec{u}}
	%+
	%\vec{\omega}\cdot  \nu \grad^{2} \vec{\omega}
	%\label{eq:enst1}
\end{align}
\end{subequations}


%%--------------------------------------------------------------------

Several balances between terms to maintain vortices are thinkable here:

%%####################################################################
%%####################################################################



\begin{eddy}[Frontal Lenses]\label{eddy:FrontalLense}
\begin{description}[noitemsep,nolistsep]
\item[large:]\hspace{50 pt}
 $\Rh$, $\Bu$, U
%\item[$\le 1$:]\hspace{62 pt}
\item[small:]\hspace{50 pt}
$\Ro$, W
\item[balance between:]
$\rnm{2}$, $\rnm{3}$ and $\rnm{5}$
%\item[significant vorticity term:]
%$A$
\end{description}
The case with strong density gradients, large current speeds and a Rossby number approaching unity is typical for the meandering tails of turbulent boundary
currents and zonal jets as in the Gulf Stream respective cyclogenesis in the atmospheric jet stream. Technically the intrathermoclinic lenses
\citep{Cushman-Roisin1990} and strong-density-gradient deep eddies e.g. \textit{meddies} fall into this group as well. With strong stratification, small
vertical displacements cause strong pressure gradients. The dynamics can be limited to some thin layer, bottom topography is of little relevance and the surface
signal might be small, or misleading.
 \end{eddy}
%###################################%


\begin{eddy}[Small Mid-Latitude Geostrophic Eddies] \label{eddy:midlat}
\begin{description}[noitemsep,nolistsep]
\item[large:]\hspace{50 pt}
 $\Rh$
\item[$\mathcal{O} 1$:]\hspace{62 pt}
$\Bu$
\item[small:]\hspace{50 pt}
$\Ro$
\item[balance between:]
$\rnm{3}$ and $\rnm{5}$
%\item[significant vorticity term:]
%none. stationary at first approximation.
\end{description}
The true geostrophic eddy with $L \sim \mathrm{L_R} \sim NH/f$.
\end{eddy}
%%####################################################################
%%####################################################################

\begin{eddy}[Large Geostrophic Gyres] \label{eddy:gyres}
\begin{description}[noitemsep,nolistsep]
%\item[large:]\hspace{50 pt}
%\item[$\mathcal{O} 1$:]\hspace{62 pt}
\item[small:]\hspace{50 pt}
$\Ro$, $\Rh$, $\Bu$
\item[balance between:]
$\rnm{3}$,   $\rnm{4}$, $\rnm{5}$ and friction
%\item[significant vorticity term:]
%$B$, $C$
\end{description}
The large-scale wind-driven ocean gyres. These can only be interpreted as an \textit{eddy} from the Reynolds-averaged large-scale perspective. The motion is strongly $f/H$-contour guided and the $\beta$-effect is immediately apparent in their strong western boundary intensification.
\end{eddy}


%%####################################################################
%%####################################################################
\begin{eddy}[the \textit{Rossby-wave}-eddy] \label{eddy:rossbywave}
\begin{description}[noitemsep,nolistsep]
\item[large:]\hspace{50 pt}
L
\item[$\mathcal{O} 1$:]\hspace{62 pt}
$\Bu$
\item[small:]\hspace{50 pt}
$\Ro$ $\Rh$
\item[balance between:]
$\rnm{3}$,$\rnm{4}$ and $\rnm{5}$
%\item[significant vorticity term:]
%$B$
\end{description}
In low latitudes quasi-geostrophy and hence a small Rossby number demand large
$L$ and/or small $U$. The pressure gradients and hence surface elevation is
small. Due to the large meridional extent, slow time-scale and strong
$f(y)$-gradient, particles moving north or south experience strong changes in
planetary vorticity. So much so, that in this regime geostrophic eddies and
Rossby waves are no longer clearly separable phenomena.
\end{eddy}
%%####################################################################
%%####################################################################


\begin{eddy}[\textit{bonus: tornado}] \label{eddy:tornado}
\begin{description}[noitemsep,nolistsep]
\item[large:]\hspace{50 pt}
$U$, $g'$, $\mathrm{L_R}$ ,$\Ro$, $\Bu$, $\Rh$
\item[small:]\hspace{50 pt}
$L$
\item[balance between:]
$\rnm{2}$ and $\rnm{5}$
\item[significant vorticity term:]
$A$ and friction (not considered here)
\end{description}\vspace{2pt}
This case isn't really applicable to the ocean except for maybe the tropics where $f$ vanishes (but $\nu$ would become relevant) or on small scales in areas of strong tidal currents in combination with bathymetry \ie \textit{tidal bores} etc . In this case a pressure force would have to be balanced by a centrifugal force alone (\eg \textit{bathtub}).
\end{eddy}






