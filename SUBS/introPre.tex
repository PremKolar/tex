\newthought{The main purpose }~of this study is to investigate the dynamics of mesoscale ocean eddies on a global scale, \ie to provide a statistical census on horizontal scale, lifetime and zonal drift speeds.
%\newthought{The main purpose }~of this study is to investigate the dynamics, in particular horizontal scales and zonal translational speeds, of geostrophic mesoscale ocean eddies on a global scale.
By virtue of the geostrophic character of the scales of concern, such vortices implicate a local upheaval/depression of density surfaces, usually also including the sea surface\footnote{As in theory, baroclinic eddies have most of their energy in the first (surface-intensified) baroclinic mode \citep{olbers2012ocean}.}.
\begin{wrapfigure}{r}{.6\textwidth}
\includegraphics[width=0.6\textwidth]{scales}
\caption{Resolutions for model vs satellite. Modified version from \citet{olbers2012ocean}.}
\label{fig:scales}
\end{wrapfigure}
The resultant \textit{hills} and \textit{valleys} in surface anomaly can be resolved by combining multiple satellite-altimetry signals (see \cref{fig:scales}). One motivation of this study is to investigate whether the resolutions in space and time of such altimeter-derived products suffice to successfully track individual eddies over long periods of time and to precisely determine their horizontal extent and drift speed. The detection/tracking/analyzing procedure of individual eddies is done globally via an automated parallelized computer-program. To analyze the effects of different time/space-resolutions, a finer-grid \SSH-product of a modern ocean-circulation model is subjected to the algorithm as well.
