

\newthought{The main purpose }~of this study is to investigate the dynamics of geostrophic mesoscale ocean eddies on a global scale.
%\newthought{The main purpose }~of this study is to investigate the dynamics, in particular horizontal scales and zonal translational speeds, of geostrophic mesoscale ocean eddies on a global scale.
By virtue of the geostrophic character of the scales of concern, such vortices implicate a local upheaval/depression of density surfaces which usually \footnote{As in theory, baroclinic eddies have most of their energy in the first (surface-intensified) baroclinic mode \citep{olbers2012ocean}.} also includes the sea surface.
\begin{wrapfigure}{r}{.6\textwidth}
\includegraphics[width=0.6\textwidth]{scales}
\caption{Resolutions for model vs satellite. Modified version from \citet{olbers2012ocean}.}
\label{fig:scales}
\end{wrapfigure}
 The resultant \textit{hills} and \textit{valleys} in sea surface anomaly can be resolved by combining multiple satellite-altimetry signals (see \cref{fig:scales}). One motivation of this study is to investigate whether the resolutions in space and time of such altimeter-derived products suffices to successfully track individual eddies over long periods of time and to precisely determine their horizontal extent. The detection/tracking/analyzing procedure of individual eddies is executed globally via an automated parallelized computer-program. 
To analyze the effects of different time/space-resolutions, a finer-grid SSH-product of a modern ocean-circulation model is subjected to the algorithm as well.

\newthought{Due}~to the inherently technical character of the matter, large parts of this text are dedicated to details of the algorithm \footnote{see the \cref{chap:algorithm}.}. Oceanographic results are treated in the \href{chap:results}{results}- and \href{chap:discussion}{discussion}-chapters. This chapter discusses the physics of mesoscale geostrophic turbulence and introduces a handful of relevant historical papers. Since focus is on horizontal scales, translational speeds and the comparison of results between the \AVI-altimetry product and SSH-data from the \POP~ocean model, sections generally focus on either of these three topics.

%\newthought{The main purpose }  of this study is to create a computer program that is able to \textbf{detect}, \textbf{track} and \textbf{analyze} mesoscale ocean eddies via their surface signal in sea-surface-height (SSH).
